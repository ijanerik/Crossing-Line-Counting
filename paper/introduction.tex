\section{Introduction}
In a world with increasing accessibility to information, it is important to summarize this information into useful information for the users.  In this case we have access to a huge amount of surveillance camera's which are used during busy events. To monitor all these camera's it requires a lot of manual watching what is happening on those camera's. There has a lot of research been done on reducing the amount of manual watching and translating the information into more actionable and measurable tasks.

The research can be ranging from quite low level to more high level. On a low level we have Crowd Counting which is to measure how how many individuals are present in an certain frame. As well as which direction a person is going, Crowd Direction. On a mid level we try to extract information over a time frame of the frame, Line of Interest is a good example of this. With Line of Interest the goal is to count how many people crossed the line in a time frame. An example of this is counting how many people went inside a stadium and went out. Lastly on a high level we want to make decisions based on both these high and mid level information, such as finding jam's inside a crowd.

In this thesis we will mostly focus on Line of Interest. Line of Interest is an area which hasn't had a lot of improvements since 2016, but there were still a lot improvements around the building blocks of LOI. Which is an excelent reason to check again the current LOI state-of-the-art and try to find ways to improve the SOTA with more up to date methods.

In the newer methods which we will be focusing on, there are two main components for Line of Interest. Which is Crowd Counting and Crowd Direction. Combining these two provides a estimation who crossed the line and who didn't.

\begin{tikzpicture}[>=latex']
    \tikzset{block/.style= {draw, rectangle, align=center,minimum width=2cm,minimum height=1cm, inner sep=0.2cm, fill=gray!20},
    rblock/.style={draw, shape=rectangle,rounded corners=1.5em,align=center,minimum width=2cm,minimum height=1cm, inner sep=0.2cm},
    }
    \node [rblock]  (start) {Input};
    \node [block, below right =1.7cm and 0.7cm of start] (countnet) {Crowd Counting\\Network (CSRNet)};
    \node [block, below left = 1.7cm and 0.7cm of start] (flownet) {Optical Flow\\Estimator (DDFlow)};
    \node [block, below = 4.3cm of start] (linenet) {Crossing Line\\Network};
    \node [rblock, below right =0.5cm and -0.5cm of linenet] (crossright) {Number\\Crossing Right};
    \node [rblock, below left = 0.5cm and -0.5cm of linenet] (crossleft) {Number\\Crossing Left};
    %% Coordinate on left and middle

% paths
    \path[draw,->, every node/.style={sloped,anchor=south,auto=false, font=\scriptsize}]
                (start) edge node {Frame A} (countnet)
                (start) edge node {Frame pair} (flownet)
                (countnet) edge node {Density Map} (linenet)
                (flownet) edge node {Flow Map} (linenet)
                (linenet) edge (crossright)
                (linenet) edge (crossleft)
                ;
\end{tikzpicture}

\subsubsection{Why unsupervised}
For Crowd Counting there are a lot of datasets available and if new data is provided, it is pretty easy to label these and update the model. For Crowd Direction this is a bit more tricky. Older methods use conventional methods which used handcrafted features which predicted pretty well the direction in situations which were pretty standard, but completely failed in situations which were different from the standard. With the introduction of convolutional neural networks in the field this changed. These made it possible to make vastly more robust filters, but with cost of requiring a lot of labeled data (Because supervised learning). All approaches in the earlier papers had the access to huge amount of labeled data, which made this a suitable solution. In situations with not a lot of available data this makes it a lot harder.
The supervised models for Crowd Direction are largely based on general Flow Estimation models which try to predict all the pixels in a frame. More recently there appear methods to make the supervised Flow Estimation models unsupervised. This makes it possible to accurately predict the direction of pixels without the need for huge amounts of labeled data.
