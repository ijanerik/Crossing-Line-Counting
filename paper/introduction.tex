\chapter{Introduction}
\section{Introduction}
In recent years the amount of surveillance camera's has increased immensely to a point that it is hard to supervise them all manually by humans. Since the increased use of surveillance cameras a lot of research is done to automate information extraction from those cameras \cite{Sreenu2019}.

A widely researched area for extracting information from camera's is Crowd Counting \cite{Chan2008, wang2020nwpu, li2018csrnet, Fang2019, Liu2019}. In Crowd Counting the amount of pedestrians present in the frame is counted. Where pedestrians in low density areas can be easily counted by general object recognition, higher density area's need specialized methods to accurately count the amount of pedestrians \cite{Zhang2016} using density maps.

While Crowd Counting only focusses on counting the exact amount of pedestrians in a frame, it doesn't take into account the amount of pedestrians walked by over time. This is no issue when camera's are present in the whole area of interest, however with large areas the amount of required camera's increases quickly, this can become an issue. In for example a shop it would be much more convenient to count the amount of customers inside the shop by only tracking the customers walking inside and outside the shop, rather than having cameras in the whole shop.

This research area of Crowd Line Crossing is much less researched \cite{Ma2013, leibe_crossing-line_2016, zheng_cross-line_2019}. By estimation both a Flow Map and a Crowd Density Map (Crowd Counting) the flow of pedestrians can be obtained and the amount of people entering and exiting an area can be counted.

In early papers on Crowd Crossing Counting prediction was done using keypoint extraction and feeding the keypoints to a regression model \cite{ma_counting_2016, Ma2013}. More recently the introduction of Convolutional Neural Networks was made into the field of both Crowd Counting \cite{Zhang2016, Liu2019, li2018csrnet, wang2020nwpu}, Flow Estimation \cite{sun_pwc-net_2018, Dosovitskiy2015} and Unsupervised Flow Estimation \cite{Yu2016, Janai2018, liu_ddflow_2019, liu_selflow_2019}. Which sparked the research in those fields.

\todo{Reframe to new life to this field}
In Crowd Crossing Counting the amount of research done using Neural Networks is limited \cite{leibe_crossing-line_2016, cao_large_2015}. New research in both Crowd Counting and Flow estimation provides lots of new opportunities to improve the State-of-the-Art of Line Crossing. New research also adds some new challenges, which we try to solve as well in this thesis.

\section{Contributions}
The latest Neural Networks for Crowd Crossing Counting \cite{leibe_crossing-line_2016} learns both Crowd Counting and Flow Estimation in a supervised way. However the labelling process of both these methods are a time consuming method, especially labelling the images for Flow Estimation. To reduce the labour intensive task of labelling, a new method is introduced to use and train unsupervised Flow Estimation during Line Crossing Counting.

Additionally a multi-headed model is proposed which trains the Crowd Counting in a supervised matter and the Flow Estimation in an unsupervised matter. The model utilizes a shared encoder, while two distinct decoders are used to predict each an individual task.

To further utilize the availability of both the predictions a flow enhanced model is proposed. This model uses the Flow Estimation to further enhance the Crowd Counting prediction.

Due to dated datasets three new datasets are proposed, which are used for comparing the proposed networks against several strong existing baselines. Two  datasets contain pedestrians, which are labeled with a self-build labeler. The third datasets contains cars on the road to show the generality of the system.

Lastly thorough research is done on the usability of the presented system in real world scenario's. Impact on video frame rate are tested and the processing time is compared.


\section{Thesis outline}
The rest of this thesis is divided into the next chapters:

 \begin{itemize}
 	\item \textbf{Background}, explains several fields to understand the starting point for this thesis.
    \item \textbf{Method}, presents the method of the proposed solution.
    \item \textbf{Implementation}, presents the hyperparameters and evaluation methods.
    \item \textbf{Datasets}, presents the used datasets and used approach to label the proposed new datasets.
    \item \textbf{Results}, discusses the results of the experiments.
    \item \textbf{Conclusion}, wraps it up and summarizes what we can conclude.
 \end{itemize}
